\documentclass[
    a4paper,
    doc, %jou, %man
    natbib
]{apa6}

\usepackage[british]{babel}
\usepackage[utf8]{inputenc}
\usepackage{epstopdf}
\usepackage{csquotes}
\usepackage[hidelinks]{hyperref}
\usepackage{natbib}


%%%%%%%%%%%%%%%%%%%%%%%%%%%%%%%%%%%%%%%%%%%%%%%%%%%

\title{An excellent study that you should definitely read}
\shorttitle{Your student name and number}
\author{Maxine Mustermann}
\affiliation{Centre for Cognitive Science, Technical University of Darmstadt}

\abstract{This is where your abstract goes.}

\keywords{vision, illusory contours, psychophysics}

\authornote{Submitted as assessment for the course ``Wahrnehmen'',  SoSe 2023.}

\begin{document}

\maketitle

This is where the introduction goes. In APA style, the introduction does not have a separate heading.

In APA style, you will often refer to citations in parenthesis at the end of the sentence to provide the reader some reference for the immediately preceding statement, like this \citep{ringachSpatialTemporalProperties1996}.

Sometimes you may want to refer to authors' names in text.
To do this in APA style, the authors' names should come in the body of the text, with the publication year in parentheses, as in \citet{ringachSpatialTemporalProperties1996}.
This automatically includes the authors' names as appropriate for APA format, with the publication year in parentheses at the end.
There are different abbreviations when you have three or more authors. 
I strongly encourage you to use a reference manager to do this automatically, rather than formatting references yourself.

\section{Methods}

The methods section can have subsections or subsubsections as appropriate for the study. I have suggested some here.

\subsection{Participants}
Describe your participants, how they were recruited, demographic information.

\subsection{Equipment}

Describe the equipment you used to conduct the experiment, including hardware and software relevant to the presentation of stimuli and collection of data.
You should also describe elements of the physical setup. 
How far from the screen was the participant (how large were the stimuli on the retina)?
Data analysis software can come under ``Data Analysis'' below.
If each participant used different hardware to do the experiment, this should be described! 

\subsection{Stimuli and procedure}

Describe how the stimuli were created and their properties.

Describe what the participant(s) did in the experiment (their task).

Describe what one trial was like.

Describe how many trials were performed, how they were blocked or randomly arranged, etc.

\subsection{Data analysis}

Describe how you analysed the data. What software (versions?) was used? 
What functions / equations did you implement?
Describe the equations and related aspects in sufficient detail that someone familiar with the field understands what you've done.

You should not include actual code here. 


\section{Results}

Describe the results of Experiment 1. Refer to data figures, but provide more narrative in text. Try to keep interpretation ("what it means") to a minimum -- that goes in the discussion.

\section{Discussion}

Your discussion section goes here.

\bibliography{bibliography.bib}

\end{document}